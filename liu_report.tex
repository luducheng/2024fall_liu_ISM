\documentclass[12pt,a4paper]{article}

\usepackage[rmargin=1.2in,lmargin=1.2in,tmargin=.8in,bmargin=.8in]{geometry}
\usepackage{indentfirst}

% fonts
% \usepackage{mathptmx}
\usepackage{fontspec}
\setmainfont{Times New Roman}

% images, table and colors
\usepackage[dvipsnames]{xcolor}
\usepackage{graphicx} 
\graphicspath{ {./figures/}, {/Users/luducheng/Documents/templates/logos/OBSPM/LERMA/RGB - RVB/Noir/}} 
\usepackage{wrapfig}

\usepackage{tabularx, makecell, booktabs}
\usepackage{caption}
\usepackage{enumitem}
\usepackage[normalem]{ulem}

% math support
\usepackage{amsmath}
\usepackage{amssymb,stackengine}
\newcommand\varlesssim{\mathrel{\ensurestackMath{%
  \stackengine{-.4ex}{<}{\rotatebox{-25}{$\sim$}}{U}{r}{F}{T}{S}}}}
\newcommand\vargtrsim{\mathrel{\ensurestackMath{%
  \stackengine{-.4ex}{>}{\rotatebox{25}{$\sim$}}{U}{l}{F}{T}{S}}}}
\usepackage{siunitx}
\DeclareSIUnit{\erg}{erg}
\DeclareSIUnit{\rate}{\erg \per \centi \metre \cubed \per \second}
\usepackage[version=4]{mhchem}

% for code display
\usepackage{minted}

% citations
\usepackage[colorlinks=true,allcolors=blue]{hyperref}
\usepackage[backend=biber,citestyle=authoryear-icomp,bibstyle=authoryear,hyperref=true,isbn=false,url=false,eprint=false,date=year,maxcitenames=2,giveninits=true,uniquename=init,uniquelist=minyear,sorting=nyt]{biblatex}
\DeclareNameAlias{default}{family-given}
\DeclareNameAlias{sortname}{family-given}
\addbibresource{./pdr_references.bib}
\input{/Users/luducheng/Documents/templates/asstex} % to compile on my computer
\renewcommand*{\thefootnote}{(\arabic{footnote})}
% \input{asstex}

%%%%% define your own command %%%%%
\newcommand{\mr}{\mathrm}
% derivatives
\newcommand{\lfird}[2][]{\mathrm{d}#1/\mathrm{d}#2} 
\newcommand{\fird}[2][]{\frac{\mathrm{d}#1}{\mathrm{d}#2}} 
\newcommand{\secd}[2][]{\frac{\mathrm{d}^2#1}{\mathrm{d}#2^2}}
\newcommand{\pfird}[2][]{\frac{\partial#1}{\partial#2}} 
\newcommand{\pfirdat}[3][1]{\left(\frac{\partial#1}{\partial#2}\right)_{\!\!\!#3}} 
\newcommand{\dd}[1]{\mathrm{d}#1}

\newcommand{\mdpdr}{\mintinline{latex}{MeudonPDR} code}
\newcommand{\qt}[1]{\textcolor{red}{#1}}

\title{Explaining Spectral Line Profiles in the Horsehead Nebula Using Cloud Surface Curvature}
\author{Ducheng Lu}
\date{Jan 2025}

\begin{document}

\thispagestyle{empty}
\begin{figure}
  \raggedright
  $\vcenter{\hbox{\includegraphics[width=.75\textwidth,keepaspectratio]{Observatoire_de_Paris-CoMarquageLERMA-RGB-Noir_sideral.pdf}}}$
  \hfill
\end{figure}

\vspace*{7em}
\begin{center}
    \rule{\textwidth}{2pt}
    \vskip3em
    \LARGE{Explaining Spectral Line Profiles in the Horsehead Nebula Using Cloud Surface Curvature}
    \vskip1em
    \rule{\textwidth}{2pt}
\end{center}
\vfill
\begin{flushright}
    \large
    Student\\
    Ducheng Lu\\[1em]
    Supervisors\\
    Franck Le Petit (LERMA)\\
    Emeric Bron (LERMA)
    \vskip1em
    Jan 2025
\end{flushright}
\vspace*{3em}


\newpage
\thispagestyle{empty}
\vspace*{10em}
{\hypersetup{hidelinks}\large
\tableofcontents
}

\newpage
\clearpage
\pagenumbering{arabic} 
\begin{abstract}
\normalsize

\textit{Context.} 

\textit{Aim.} 

\textit{Methods.} 

\textit{Results.} 

\end{abstract}

\begin{abstract}
\normalsize
\textit{Contexte.} 

\textit{Objectif.} 

\textit{Méthodes.} 

\textit{Résultats.}
\end{abstract}

\newpage
\section{Introduction}

The interstellar medium (ISM) is composed of gas and dust between stars in galaxies. The ISM is site of star formation and takes up around $\sim 10\%$ of total baryonic mass \qt{cite Drain2011}. In return, the radiation field produced by stars is responsible for the heating the gas and dust that cool in bright line and continuum emission. These emission lines provide information about the physical conditions inside the ISM, such as the temperature, density and chemical composition.

Models have long been developed to help us interpret these


The 1D slab geometry may not be approprioate in some cases, and the curvature of the PDR region needs to be taken into account in order to reproduce the line profiles. A common technique is to use a time-dependent hydrodynamic simulation to obtain the density and velocity fields and then to postprocess it with a PDR code to obtain the steady-state chemical abundances and thermal equilibrium gas temperature (e.g., Levrier et al. 2012).

\subsection{Photodissociation Regions (PDRs)}
Photodissociation regions (PDRs) are regions where far-ultraviolet (FUV; $\qty{6}{eV} < h\nu < \qty{13.6}{eV}$) radiation dominates the chemistry or heating processes\parencite{Tielens1985}. PDRs span a wide range of incident FUV fluxes and densities, including all neutral gas in the interstellar medium (ISM) and molecular layers where FUV radiation drives molecule formation.

\begin{figure}
    \centering
    \includegraphics[width=.7\textwidth,keepaspectratio]{figures/PDRScheme_Wolfire2022fig2.png}
    \caption{Schematic of a photodissociation region as a function of visual extinction $A_V$. Reprinted from \textcite{Wolfire2022}.}
\end{figure}
\subsection{The Horsehead Nebula}
\begin{figure}
    \centering
    \includegraphics[width=.7\textwidth,keepaspectratio]{horsehead_hst.jpg}
    \caption{\href{https://hubblesite.org/contents/media/images/2013/12/3166-Image.html?keyword=Horsehead}{Horsehead Nebula captured by the Hubble Space Telescope (HST) in 2013.} The image was created from Hubble data from proposal \href{http://archive.stsci.edu/proposal_search.php?mission=hst&id=12812}{12812}. Illustration Credit: \href{http://www.nasa.gov/}{NASA}, \href{http://www.spacetelescope.org/}{ESA}, and Z. Levay (\href{http://www.stsci.edu/}{STScI})} 
\end{figure}

% \begin{figure}
%     \centering
%     \includegraphics[width=\textwidth,keepaspectratio]{horsehead_zoomed.pdf}
%     \caption{\href{https://www.esa.int/ESA_Multimedia/Images/2024/04/Webb_captures_iconic_Horsehead_Nebula_in_unprecedented_detail}{Three views of the Horsehead Nebula}. The first image (left) features the Horsehead Nebula as seen by ESA's Euclid telescope. The second image (middle) shows the NASA/ESA Hubble Space Telescope's infrared view of the Horsehead Nebula. The third image (right) features a new view of the Horsehead Nebula from the NASA/ESA/CSA James Webb Space Telescope's NIRCam (Near-InfraRed Camera) instrument.}
% \end{figure}

\begin{figure}
    \centering
    \includegraphics[width=\textwidth,keepaspectratio]{horsehead_HernandezVera2023.pdf}
    \caption{Multiphase view of the Horsehead nebula (Barnard 33). Reprinted from \textcite{HernándezVera2023}. Left: Composite color image of the Horsehead nebula observed with the VLT (ESO). Right: Zoomed-in view of the edge of the molecular cloud, imaged with ALMA in the \ce{CO} $J = 3-2$ line emission (blue) and the \qty{0.9}{m} KPNO telescope in the H$\alpha$ line emission. The dark region represents the neutral atomic layer.} \label{fig:obsimg}
\end{figure}

In order to compare with observations, we need to convert the distance unit \unit{cm} used the in \mdpdr{} into the \unit{arcsec} unit used in observation,

\begin{equation}
    \alpha["] = \frac{d\,[\unit{pc}]}{400\,\unit{pc}} \frac{1\,\unit{arcsec}}{1\,\unit{rad}}\textbf{}
\end{equation}
where \qty{400}{pc} \parencite{Menten2007, Schlafly2014} is the distance to the horsehead nebula.

\section{Data}
The observational data used in this study (Fig.~\ref{fig:observation}) were obtained from the Heterodyne Instrument for the Far-Infrared (HIFI) \parencite{deGraauw2010} onboard the Herschel Space Observatory \parencite{Pilbratt2010}. All data are convolved to the spatial resolution of the HIFI H2O data, $38.1$ arcsec. The observed molecular species were detected at specific wavelengths corresponding to transitions listed in Table~\ref{tab:lines}. 

\begin{figure}
    \centering
    \includegraphics[width=\textwidth,keepaspectratio]{observed_lines.pdf}
    \caption{Observed line profiles of the Horsehead Nebula by Herschel HIFI. Reproduced from \qt{cite the original paper}. \qt{?The \ce{^{13}CO} lines has been removed due to bad weather condition during the observation.}} \label{fig:observation}
\end{figure}

\begin{table}[h!]
    \centering
    \begin{tabular}{cclll}
        \midrule
        \midrule
        Notation & Species & Upper Level & Lower Level & Frequency (\unit{GHz}) \\
        \midrule
        \ce{H2O}            & \ce{H2O}  & J=1, Ka=1, Kc=0 & J=1, Ka=0, Kc=1 & 557.30 \\
        \ce{C[II]}          & \ce{C+}   & 2P J=3/2 & 2P J=1/2 & 1902.59 \\
        \ce{C[I]}           & \ce{C}    & 3P J=1 & 3P J=0 & 492.02 \\
        \ce{^{12}CO} (2-1)  & \ce{CO}   & v=0, J=2 & v=0, J=1 & 230.538 \\
        \ce{^{12}CO} (4-3)  & \ce{CO}   & v=0, J=4 & v=0, J=3 & 461.041 \\
        \ce{^{13}CO} (2-1)  & \ce{^{13}CO} & v=0, J=2 & v=0, J=1 & 220.399 \\
        \ce{C^{18}O} (2-1)  & \ce{C^{18}O} & v=0, J=2 & v=0, J=1 & 219.560 \\
        \midrule
        \bottomrule
    \end{tabular}
    \caption{Parameters of transitions in the observation data used in this study.} \label{tab:lines}
\end{table}

% The observational data used in this study were obtained from the [Name of Survey or Telescope], which observed the Horsehead Nebula at millimeter and infrared wavelengths. These observations focus on the molecular gas in the nebula, particularly the abundances and temperature of key species such as \(\ce{H2}\), \(\ce{CO}\), and \(\ce{H2O}\). The observations provide high-resolution spectra and spatial data, allowing us to examine the molecular structure of the nebula at various depths.

% The data were obtained with a spatial resolution of [x] arcseconds and a spectral resolution of [y] km/s. The observed molecular species were detected at specific wavelengths corresponding to rotational transitions of \(\ce{H2}\), \(\ce{CO}\), and \(\ce{H2O}\). These observations provide valuable insights into the molecular abundances and gas temperature at different layers of the nebula.


\section{Methods}
\subsection{The \mdpdr{}}
% first, ISM models in general
% caveats? PDR annual review section 5.5
A significant heterogeneity exists among the available PDR models, which differ in their geometry, physical and chemical structures, and model parameters. \textcite{Röllig2007} suggest that the choice of a spedific code should depend on the physical and chemical processes implemented in the code, as well as the characteristics of the emission source. 

In this project, I used the \mdpdr{} \parencite{LePetit2006,Goicoechea2007,Gonzalez2008,LeBourlot2012,Bron_thesis,Bron2014,Bron2016} to simulate the Horsehead Nebula. The \mdpdr{} models a stationary one-dimensional (1D) slab of gas and dust illuminated by an ultraviolet (UV) radiation field from one or both sides (Fig.~\ref{fig:1Dgeometry}). At each iteration, the code solves the UV radiative transfer in both the continuum and lines, followed by the chemical balance, and finally the level populations and thermal balance. The code outputs the level populations, the gas temperature, and the chemical abundances as a function of the depth into the cloud, with optional output of the radiation field. The physical parameters used to model the Horsehead PDR are summarized in Table~\ref{tab:params}.

\begin{table}[h!]
    \centering
    \begin{tabular}{ll}
        \midrule
        \midrule
        Cloud size ($A_{V, \max}$\footnotemark[1]) & 40 \\
        Proton density\footnotemark[2] ($n_H$) & \qtyrange[range-units=single,range-phrase=~--~]{3e4}{3e6}{cm^{-3}}\\
        Pressure\footnotemark[2] ($P$) & \qtyrange[range-units=single,range-phrase=~--~]{1e6}{1e7}{K\,cm^{-3}} \\
        ISRF & shape: Mathis\footnotemark[3], geometry: beam\_isot\footnotemark[4] \\
        ISRF scaling factor & $G_0^\mr{obs} = 100,\ G_0^\mr{back} = 0.04$\footnotemark[5]\\
        UV radiative transfer method & FGK approximation, or\\
        & exact \ce{H2} self- and mutual shielding\footnotemark[6] \\
        Turbulent velocity dispersion & \qty{2}{\km\per\second}\footnotemark[7] \\
        Extinction Curve & HD38087\footnotemark[8]\\
        $R_V = A_V / E(B-V)$ & $5.50$ \\
        $C_D = N_H / E(B-V)$ & $1.57\times 10^{22}$\\
        \bottomrule
    \end{tabular}
    \caption{Key physical parameters used to model the Horsehead PDR in the \mdpdr{}.\textsuperscript{1}Visual extinction, $A_V \equiv 2.5\log_{10}(I_\mr{V}^0/I_\mr{V}^\mr{obs})$.  \textsuperscript{2}Proton density is used in constant density models, and pressure is used in constant pressure models(see Section~\ref{sec:cstnp}). \textsuperscript{3}\textcite{Mathis1983}. \textsuperscript{4}Perpendicular on observer side, isotropic on back side. \textsuperscript{5}In Habing units, as defined in \textcite{LePetit2006}. \textsuperscript{6} See Section.~\ref{sec:exactrt}. \textsuperscript{7}Only used in Doppler line broadening. \textsuperscript{8}\textcite{Fitzpatrick1990}. Other parameters use default values. More descriptions can be found in the \href{https://ism.obspm.fr/files/PDRDocumentation/PDRDoc7.pdf}{\mintinline{latex}{MeudonPDR}} documentation.} \label{tab:params}
\end{table}

\begin{figure}
    \centering
    % \includegraphics[width=.6\textwidth,keepaspectratio]{slab_geometry.png}
    % \caption{Scheme of the slab geometry of the \mdpdr{}. Reprinted from \textcite{LePetit2006}.} \label{fig:1Dgeometry}
    \includegraphics[width=.7\textwidth,keepaspectratio]{schemePDR.pdf}
    \caption{Scheme of the slab geometry in \mdpdr{}. $A_V = 0$ at the observer side and increases toward the back side, reaching $A_V = A_{V, \max}$, which controls the size of the cloud. Reprinted from Fig.~3.1 in \href{https://ism.obspm.fr/files/PDRDocumentation/PDRDoc7.pdf}{\mintinline{latex}{MeudonPDR}} documentation.} \label{fig:1Dgeometry}
\end{figure}

\subsubsection{Constant Pressure vs. Constant Density} \label{sec:cstnp}

The density structure of a PDR model, whether it assumes constant density, constant pressure, or a user-defined density profile, can significantly influence the simulation results \parencite{Wolfire2022}. As shown in Fig.~\ref{fig:cmp_cstp_cstn} and Fig.~\ref{fig:cmp_cstp_cstn_arcsec}, I compare the cloud structure computed with constant pressure and constant density assumptions, illustrating the differences in the resulting PDR structures. The values $n_H = \qty{3e5}{cm^{-3}}$ for the constant density model and $P = \qty{5e6}{K\,cm^{-3}}$ for the constant pressure model are based on \textcite{Maillard2023}. In the constant pressure model, the transition from atomic to molecular hydrogen occurs at higher visual extinction (Fig.~\ref{fig:cmp_cstp_cstn}), and the total physical thickness of the cloud is larger (Fig.~\ref{fig:cmp_cstp_cstn_arcsec}). The temperature profile in the constant pressure model is also more gradual. Additional comparisons between various constant density models and constant pressure models are provided \qt{in the appendix}.

Observations of the Horsehead Nebula reveal a steep density gradient in the PDRs \parencite{Habart2005,Guzmán2011}. \textcite{HernándezVera2023} showed that constant density models fail to reproduce the observed structures, and neither do previously proposed density profile prescriptions. Furthermore, recent observations from ALMA and Herschel indicate that the warm layer of PDRs is indeed isobaric, with by relatively high thermal pressures \parencite{Marconi1998,Goicoechea2016,Joblin2018,Wu2018,Bron2018,Maillard2021}. Based on these findings, I adopt a constant pressure model with $P = \qty{5e6}{K\,cm^{-3}}$ for the subsequent analysis.

\begin{figure}[ht]
    \centering
    \includegraphics[width=.49\textwidth,keepaspectratio]{struct_nH3e5.pdf}
    \includegraphics[width=.49\textwidth,keepaspectratio]{struct_P5e6.pdf}
    \caption{Comparison of the cloud structure computed with constant density $n_H = \qty{3e5}{cm^{-3}}$ (left) and constant pressure $P = \qty{5e6}{K\,cm^{-3}}$ (right), with a shared legend displayed in the left plot. All other parameters are the same as those listed in Table.~\ref{tab:params}.} \label{fig:cmp_cstp_cstn}
\end{figure}

\begin{figure}[hb]
    \centering
    \includegraphics[width=.49\textwidth,keepaspectratio]{struct_nH3e5_arcsec.pdf}
    \includegraphics[width=.49\textwidth,keepaspectratio]{struct_P5e6_arcsec.pdf}
    \caption{Similar to Fig.~\ref{fig:cmp_cstp_cstn}, but using physical units, with arcseconds as the distance unit and the densities represented by their actual values (not normalized by proton density).} \label{fig:cmp_cstp_cstn_arcsec}
\end{figure}

\subsubsection{Models with Exact Radiative Transfer of \texorpdfstring{\ce{H2}}{H2}} \label{sec:exactrt}

There are two options available in the \mdpdr{} for treating radiative transfer in the UV. The first is the FGK approximation \parencite{Federman1979}, in which the self-shielding of \ce{H} and \ce{H2} molecules is treated approximately. Self-shielding occurs when molecules absorb radiation in their own spectral lines, reducing the flux available to penetrate deeper into the cloud. However, the FGK approximation neglects mutual shielding, where the overlap and interaction of absorption lines between different species or multiple lines of the same species further attenuate the radiation field.

\begin{figure}[ht]
    \centering
    \includegraphics[width=.92\textwidth,keepaspectratio]{spectra_fgkh2_1.pdf}
    % \includegraphics[width=.88\textwidth,keepaspectratio]{spectra_fgkh2_2.pdf}
    \caption{Comparison of the local energy density of the radiation field from $1074~\text{\AA}$ to $1080~\text{\AA}$ at visual extinction depths $A_V = 0.3$ (blue), $A_V = 2$ (orange), and $A_V = 5$ (green), for models computed using the FGK approximation (dashed lines) and the exact radiative transfer method (solid lines) including the 20 lowest \ce{H2} absorption lines.} \label{fig:cmpH2rtspectra}
\end{figure}

The second, more accurate approach is to solve the radiative transfer exactly. This method explicitly includes mutual shielding for a given number of energy levels of \ce{H}, \ce{H2}, \ce{D}, \ce{HD}, \ce{CO}, \ce{^{13}CO}, and \ce{C^{18}O}, as detailed in \textcite{Goicoechea2007,Gonzalez2008}. For higher energy levels, the FGK approximation is retained, as their contributions are negligible due to their low populations.

Fig.\ref{fig:cmpH2rtspectra} compares spectra from $1074~\text{\AA}$ to $1080~\text{\AA}$ at various visual extinction values for models computed using the FGK approximation and the exact radiative transfer method, demonstrating pronounced attenuation when applying the exact method of \ce{H2} radiative transfer. Although computationally more intensive, the exact method enables a more accurate treatment of the UV radiation field, which subsequently influences the PDR structure. For instance, the exact method leads to increased attenuation of the radiation field, shifting the \ce{H}/\ce{H2} transition layer to lower extinction depths \parencite{Goicoechea2007}, as illustrated in Fig.\ref{fig:cmpH2rt}. Furthermore, the enhanced attenuation from mutual shielding results in a lower temperature profile. To balance accuracy and computational efficiency, my model applies exact radiative transfer only to the 20 lowest energy levels of \ce{H2}, which are the primary contributors to the attenuation. \qt{how to explain the second bump occuring in the both cases?}

\begin{figure}[ht]
    \centering
    \includegraphics[width=.51\textwidth,keepaspectratio]{cmpH2rt_H_O.pdf}
    \includegraphics[width=.48\textwidth,keepaspectratio]{cmpH2rt_CO_H2O.pdf}
    \caption{Comparison of temperature profile and abundances for models computed with (solid) and without (dashed) exact radiative transfer (RT) of \ce{H2}: \ce{H} species and \ce{O} (left), and \ce{C} species, \ce{CO}, and \ce{H2O} (right). } \label{fig:cmpH2rt}
\end{figure}

\subsubsection{Models with Surface Chemistry}
In the ISM, direct gas-phase formation of molecules is very inefficient. Instead, molecules are formed on the surfaces of dust grains, which act as catalysts by providing a surface for adsorbed atoms to meet and react. The grains also absorb the excess energy released by the formation process, preventing dissociation that would otherwise occur in the gas-phase formation. The formation of \ce{H2} is a key demonstration of the importance of this process, as the gas-phase formation rate of \ce{H2} is much lower than the rate required to explain the observed abundance of \ce{H2} in the ISM \parencite{Gould1963,Hollenbach1971}. In addition to chemical reactions, dust grains also play a role in the sublimation and freeze-out of molecules, leading to “jumps” or “drops” in molecular profiles at specific temperatures \parencite{Herbst2009}.

\begin{figure}[hb]
    \centering
    \includegraphics[width=.52\textwidth,keepaspectratio]{cmpsurfb_H_O.pdf}
    \includegraphics[width=.47\textwidth,keepaspectratio]{cmpsurfb_CO_H2O.pdf}
    \caption{Comparison of temperature profile and abundances for models computed with (soid lines) and without (dashed lines) surface chemistry: \ce{H} species and \ce{O} (left), and \ce{C} species, \ce{CO}, and \ce{H2O} (right).} \label{fig:cmpsurfb}
\end{figure}

In Fig.~\ref{fig:cmpsurfb}, I compare the PDR models computed with and without surface chemistry. The inclusion of surface reactions leads to an earlier rise in the abundances of $n(\ce{H2})$, $n(\ce{CO})$ and $n(\ce{H2O})$ due to enhanced production on dust grain surfaces. At greater depths within the cloud, the abundances of almost all molecules decrease in the model with surface chemistry because of the freeze-out of molecules onto the dust grains. Grains also reduce the gas temperature, as they absorb the energy released during molecule formation, as shown in the left panel. Therefore, including surface chemistry in PDR models is crucial for accurately describing the molecular abundances and the gas temperature.

To summarize the model comparisons in this section, I will use the constant pressure model with exact radiative transfer of \ce{H2} and surface chemistry to consider the curvature of the cloud's surface in the following sections.

\subsection{Column Density in a Spherical Geometry}
\begin{wrapfigure}{r}{.32\textwidth}
    \centering
    \includegraphics[width=.25\textwidth]{sphere_geometry_2LoS.pdf}
    \caption{Scheme of the spherical, plane-parallel geometry. Curvature exaggerated for clarity.} \label{fig:geometry}
\end{wrapfigure}

% \begin{figure}
%     \centering
%     \raisebox{-0.5\height}{\includegraphics[width=.69\textwidth,keepaspectratio]{figures/slab_geometry.png}}
%     \hfill
%     \raisebox{-0.5\height}{\includegraphics[width=.24\textwidth,keepaspectratio]{figures/sphere_geometry_2LoS.pdf}}
%     \caption{Left: scheme of the slab geometry of the \mdpdr{}. Reprinted from \textcite{LePetit2006} Right: scheme of the plane-parallel geometry of the PDR wrapper.} \label{fig:geometry}
% \end{figure}

To model the curvature of the cloud, I treat the PDR region as the outermost shell of a fictitious spherical cloud, assuming plane-parallel geometry, as shown in Fig.~\ref{fig:geometry}. This approximation is valid when the curvature radius of the cloud's surface is relatively small compared to the radius of the cloud. For the models in this study, the physical depth of the PDR region is approximately $\qty{0.08}{pc}$, which is significantly smaller than the curvature radius, as seen in Fig.~\ref{fig:obsimg}, thereby justifying the use of plane-parallel geometry. Under this assumption, I consider all physical variables at a given distance from the cloud's surface to be equal to those computed by the \mdpdr{} model at the same distance from the cloud's edge.

The radius $R$ of the fictitious spherical cloud is provided as an input parameter to the wrapper code. A line of sight (LoS) is defined by its impact parameter $b$, which measures the perpendicular distance from the LoS to the center of the cloud. From the output of the \mdpdr{}, the level number densities $n_X(d)$ are obtained as a function of the depth $d$ into the cloud. In plane-parallel geometry, $n_X(d)$ gives the number density at a point located at a depth $d$ from the spherical surface. 

The column density along a given LoS is determined by interpolating the number density at each depth d and then integrating along the LoS. To do this, I first compute the range of distances inside the PDR region along the LoS. The maximum half-distance $s_{\max}$ is given by:
\begin{equation}
    s_{\max} = \sqrt{R^2 - b^2},
\end{equation}
while the computation of the minimum half-distance $s_{\min}$ depends on whether the LoS penetrates beyond the PDR region (see Fig.~\ref{fig:geometry}):
\begin{equation}
    s_{\min} = \left\{\begin{array}{ll}
       0  &  \text{ if } b > R - d_\mr{PDR}, \text{ LoS 1} \\
       \sqrt{(R - d_\mr{PDR})^2 - b^2}  &  \text{ if } b < R - d_\mr{PDR}, \text{ LoS 2}
    \end{array}\right.,
\end{equation}
where $d_\mr{PDR}$ is the depth of the PDR region. Next, the distance $s$ along the LoS is converted into the depth $d$ from the cloud's surface. Using Pythagorean theorem, one can easily establish that $d = R - \sqrt{s^2 + b^2}$. The number density at each point along the LoS is then calculated as:
\begin{equation}
    n_X(s) = f(R - \sqrt{s^2 + b^2}),
\end{equation}
where $f(d)$ is the interpolated function of the level number density $n_X(d)$. Finally, the column density $N_X(b)$ for a given LoS with impact parameter $b$ is obtained by integrating the number density along the LoS. By symmetry, the total column density is twice the integral over the half-LoS: 
\begin{equation}
    N_X(b) = 2\int_{s_{\min}}^{s_{\max}} n_X(s') \dd{s'}.
\end{equation}
By considering the curvature of the cloud's surface, this column density allows for a more direct comparison with observations. However, this approach does not account for the absorption and emission processes occurring inside the cloud. For lines that are optically thick, the radiative transfer equation must be solved along the LoS to accurately reproduce the observed line profiles. This more detailed treatment will be addressed in the next section.

\subsection{Radiative transfer equation}
In the previous calculation of the column densities, I assumed that all lines were optically thin. However, some lines can be optically thick, leading to differences in the observed line profiles for various lines, as seen in Fig~\ref{fig:observation}. To account for line extinction within the cloud, I need to solve the radiative transfer equation along the LoS \parencite[see, e.g.,][ Eq.~1.67]{Rybicki1979}:
\begin{equation}
    \fird[I_\nu]{s} = A_{ul} n_u\frac{h\nu}{4\pi}\phi(\nu) +  B_{ul} n_u\frac{h\nu}{4\pi}I_\nu \phi(\nu) -  B_{lu} n_l\frac{h\nu}{4\pi}I_\nu \phi(\nu), \label{eq:rte}
\end{equation}  
where $n_u$ and $n_l$ are the number densities of the upper and lower levels of the transition, respectively, $h$ is the Planck's constant, $\nu$ is the frequency of the transition, and $\phi(\nu)$ is the line profile. For simplicity, I use the same line profile for both emission and absorption, neglecting scattering. \qt{justify?}

The coefficients $A_{ul}, B_{ul}, B_{lu}$ in Eq.~\ref{eq:rte} are the Einstein coefficients for spontaneous emission, stimulated emission, and absorption, respectively. These coefficients are intrinsic properties of the specific transition and are related through the Einstein relations \parencite{Einstein1917}:
\begin{equation}
    B_{ul} = \frac{c^2}{2 h \nu_{ul}^3} A_{ul},\, g_l B_{lu} = g_u B_{ul},
\end{equation}
where $g_u$ and $g_l$ are the degeneracies of the upper and lower levels.

In PDRs, the motion of atoms and molecules results in line broadening. Thermal motion produces a Gaussian profile, while turbulent motions contribute an additional broadening component. These effects are combined in the effective Doppler width, $\Delta \nu_D$, which is then included in the Gaussian line profile \parencite[see, e.g.,][Eqs.~10.68-10.72]{Rybicki1979}:
\begin{equation}
    \phi(\nu) = \frac{1}{\sigma_\nu \sqrt{2 \pi}}\exp\left(-\frac{(\nu - \nu_0)^2}{2 \sigma_\nu^2}\right),
\end{equation}
where
\begin{equation}
    \sigma_\nu = \frac{\sqrt{2}}{2}\Delta \nu_D, \quad \Delta \nu_D = \frac{\nu_0}{c}\sqrt{\frac{2 k T}{m} + v_\mr{turb}^2}.
\end{equation}
Here, $\nu_0$ is the central frequency of the transition, $c$ is the speed of light, $T$ is the gas temperature, $m$ is the mass of the emitting species, and $v_\mr{turb} = \qty{2}{\km\per\second}$ is the turbulent velocity, as introduced in Table.~\ref{tab:params}.

The line profile is normalized so that $\int_0^{\infty} \phi(\nu) \dd{\nu} = 1$. To ensure computational efficiency while maintaining a good representation of the profile, I truncate the line profile at $5\sigma$. Therefore, in practice, the normalization becomes:
\begin{equation}
    \int_{\nu_0 - 5\sigma}^{\nu_0 + 5\sigma}\phi(\nu) \dd{\nu} = 1.
\end{equation}
\qt{if have time, add a figure to demonstrate that this is reasonable for the lines concerned?}

To solve the radiative transfer equation, I also need a background intensity $I_0$. For this, I use the isotropic specific intensity on the observer side, as provided by the \mdpdr{} in the file \mintinline{python}{_IncRadField.dat}. It is important to note that the current implementation is valid only for isotropic radiation; handling more general forms of background specific intensity would require additional modifications. \qt{add explanation why?}

The external radiation field in the \mintinline{python}{IncRadField.dat} file is provided in units of \unit{erg\per\centi\meter\squared\per\second\per\steradian\per\angstrom}. However, to solve the radiative transfer equation, I need the specific intensity in units of \unit{erg\per\centi\meter\squared\per\second\per\steradian\per\Hz}. By energy conservation, $I_\nu |\dd{\nu}| = I_\lambda |\dd{\lambda}|$, and using $c = \lambda\nu$, I can derive the differential relation $\dd{\lambda} = -c/\nu^2 \dd{\nu}$. This allows me to convert between the two sets of units as follows:
\begin{equation}
    I_\nu = I_\lambda \left|\frac{\dd{\lambda}}{\dd{\nu}}\right| = I_\lambda \frac{c}{\nu^2} = I_\lambda \frac{\lambda^2}{c}.
\end{equation}

Thus far, we have focused on improving the modeling of the physical processes in PDRs. However, the observed spectra are inevitably influenced by the resolution of the instrument used for observation. The instrumental resolution effectively smooths out the line profile, broadening the observed lines and altering their shapes. Consequently, it is essential to account for the instrument's resolution in the modeling process. In the following section, I will discuss the convolution process and how it modifies the observed spectra.

\subsection{Convolution}

\section{Results and Discussion}
\subsection{Cloud Surface Curvature}
\subsection{Line Profiles}

\section{Conclusions}

\newpage
\printbibliography

\newpage
\appendix

\end{document}